\documentclass[]{article}


\usepackage{imports}
\usepackage{enumitem}

\title{Solutions 10}
\author{Nickel Paulsen}

\begin{document}
\maketitle



\begin{definition*}[associative algebra]
   Let \(R\) be a commutative ring, then \(A\) is called an \underline{associative algebra over \(R\)} or \underline{\(R\)-algebra} if \(A\) is an \(R\)-modul and a ring 
   such that scalars in \(R\) commute:
   \begin{align*}
     r ( a b) = (r a) b = a(rb) \ , \quad r \in R, a,b \in A
   \end{align*}
\end{definition*}

In the following let \(W\) be the Weyl group of a \(BN\)-pair \(B,N \subset G\) with generating set \(S \subset W\) which only contains elements
of order \(2\). Furthermore let \(\mathscr{H}\) be the \(R:=\Z[v,v^{-1}]\)-algebra with basis \(\{\widetilde{T_w} | w \in W\}\) and multiplication given by
\begin{align*}
    \widetilde{T}_{s} \widetilde{T}_w =
    \begin{cases}
        \widetilde{T}_{s w}& \text{if} \quad l(s w) = l(w) +1 \\
        \widetilde{T}_{s w}+(v-v^{-1})\widetilde{T}_w & \text{if} \quad l(s w)= l(w)-1 \\
    \end{cases}
\end{align*}

\begin{exercise}
    The map \(\varphi : \mathscr{H} \rightarrow \mathscr{H}\) given by
    \begin{align*}
        \varphi\left(\sum_{w \in W}^{}a_w \widetilde{T}_w\right) = \sum_{w \in W}^{}(-1)^{l(w)}\overline{a}_w\widetilde{T}_w
    \end{align*}
    is a ring homomorphism of order 2, i.e. \(\varphi^2 = id\). Furthermore consider the algebra isomorphism
    \begin{align*}
        \delta: \mathscr{H} \rightarrow \mathscr{H}, \ T_w \mapsto (-u)^{l(w)} T_w^{-1}
    \end{align*}
    from sheet 9 exercise 5, then the compositions 
    \begin{align*}
        \varphi \circ \delta = \delta \circ \varphi
    \end{align*}
    are equal. Furthermore the map 
    \begin{align*}
        \overline{\phantom{x}}:\mathscr{H} &\rightarrow \mathscr{H}, \ \overline{\sum_{w \in W}^{}a_w \widetilde{T}_w} := \sum_{w \in W}^{} \overline{a_w} \ \widetilde{T}_w^{-1} 
    \end{align*}
    from Lemma 4.1 is given by the composition
    \begin{align*}
        \overline{\phantom{x}} = \delta \circ \varphi
    \end{align*}
\end{exercise}

\begin{proof}
    \underline{claim:} \(\varphi\) is a morphism of rings.
    We start with showing that \(\varphi\) is compatible with addition:
    \begin{enumerate}
        \item Let \(h = \sum_{w \in W}^{} a_w \widetilde{T}_w, \ h' = \sum_{w \in W}^{} b_w \widetilde{T}_w \in \mathscr{H}\), then we have
        \begin{align*}
            \varphi(h+h')= \varphi\left(\sum_{w \in W}^{} (a_w+b_w) \widetilde{T}_w\right) = \sum_{w \in W}^{} (-1)^{l(w)}(\overline{a_w}+\overline{b_w}) \widetilde{T}_w = \varphi(h)+ \varphi(h')
        \end{align*}
        \item Now we want to show compatibility with multiplication. Not first that we have
        \begin{align*}
            \varphi(a h ) = \overline{a} \varphi(h), \ a \in R, h \in \mathscr{H}
        \end{align*}
        since for an arbitrary element \(h = \sum_{w \in W}^{} a_w \widetilde{T}_w\in \mathscr{H}\) we have
        \begin{align*}
            \varphi(a h ) = \varphi\left(\sum_{w \in W}^{} a a_w \widetilde{T}_w\right) = \sum_{w \in W}^{} (-1)^{l(w)}\overline{a} \ \overline{a_w} \widetilde{T}_w = \overline{a} \varphi(h)
        \end{align*}
        \item Now as a first approximation we show:
        \begin{align*}
        \varphi(\widetilde{T}_s \widetilde{T}_w) = \varphi(\widetilde{T}_s) \varphi(\widetilde{T}_w), \ \forall s \in S, w \in W
        \end{align*}
        \begin{align*}
            \varphi(\widetilde{T}_s) \varphi(\widetilde{T}_w) &= - \widetilde{T}_s \cdot (-1)^{l(w)} \widetilde{T}_w \\
            & = (-1)^{l(w)+1} \widetilde{T}_s \widetilde{T}_w
        \end{align*}
        Now if the length of \(sw\) goes up, i.e. \(l(sw)=l(w)+1\), then
        \begin{align*}
            \varphi(\widetilde{T}_s \widetilde{T}_w) = \varphi(\widetilde{T}_{sw}) = (-1)^{l(w)+1} \widetilde{T}_{sw} = (-1)^{l(w)+1} \widetilde{T}_s \widetilde{T}_w
        \end{align*}
        And also if the lenth goes down, we have
        \begin{align*}
            \varphi(\widetilde{T}_s \widetilde{T}_w) &= \varphi(\widetilde{T}_{sw}+(v-v^{-1}))\widetilde{T}_w \\
            &=(-1)^{l(w)-1} \widetilde{T}_{sw} + (-1)^{l(w)}(v^{-1}-v)\widetilde{T}_w \\
            &=(-1)^{l(w)+1} (\widetilde{T}_{sw} + (v- v^{-1}) \widetilde{T}_w) \\
            &=(-1)^{l(w)+1} \widetilde{T}_s \widetilde{T}_w
        \end{align*}
        which shows the above equation. 
        \item Now we want to show that
        \begin{align*}
            \varphi(\widetilde{T}_s h) = \varphi(\widetilde{T}_s) \varphi(h), \ s \in S, h \in \mathscr{H}
        \end{align*}
        Let \(h = \sum_{w \in W}^{}a_w \widetilde{T}_w \in \mathscr{H}\) be an arbitrary element, then we have 
        \begin{align*}
            \varphi(\widetilde{T}_s h) &= \varphi\left(\widetilde{T}_s \sum_{w}^{} a_w \widetilde{T}_w\right) \\
            &=\varphi\left( \sum_{w}^{} a_w \widetilde{T}_s\widetilde{T}_w\right) \\
            &\overset{1.}{=} \sum_{w}^{} \varphi(a_w \widetilde{T}_s\widetilde{T}_w) \\
            &\overset{2.}{=} \sum_{w}^{} \overline{a}_w \varphi(\widetilde{T}_s\widetilde{T}_w) \\
            &\overset{3.}{=} \sum_{w}^{} \overline{a}_w \varphi(\widetilde{T}_s)\varphi(\widetilde{T}_w) \\
            &=\varphi(\widetilde{T}_s) \sum_{w}^{} \overline{a}_w \varphi(\widetilde{T}_w) \\
            &= \varphi(\widetilde{T}_s) \varphi(h)
        \end{align*}
        \item Now we want to show that
        \begin{align*}
            \varphi(\widetilde{T}_y h) = \varphi(\widetilde{T}_y) \varphi(h), \ y \in W, h \in \mathscr{H}
        \end{align*}
        Let \(s_1 \dots s_n = y \in W, s_i \in S\) be a reduced form, then we have \(\widetilde{T}_{s_1}\dots \widetilde{T}_{s_n} = \widetilde{T}_y\) and thus
        \begin{align*}
            \varphi(\widetilde{T}_y h) = \varphi(\widetilde{T}_{s_1}\dots \widetilde{T}_{s_n}h) \overset{3.}{=} \varphi(\widetilde{T}_{s_1})\dots \varphi(\widetilde{T}_{s_n})\varphi(h) \overset{3.}{=} \varphi(\widetilde{T}_{s_1}\dots \widetilde{T}_{s_n}) \varphi(h ) = \varphi(\widetilde{T}_y) \varphi(h)
        \end{align*}
    \end{enumerate}
\end{proof}

\end{document}