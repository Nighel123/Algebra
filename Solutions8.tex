\documentclass[]{article}


\usepackage{imports}
\usepackage{enumitem}

\title{Solutions 8}
\author{Nickel Paulsen}

\begin{document}
\maketitle



\begin{definition*}[associative algebra]
   Let \(R\) be a commutative ring, then \(A\) is called an \underline{associative algebra over \(R\)} or \underline{\(R\)-algebra} if \(A\) is an \(R\)-modul and a ring 
   such that scalars in \(R\) commute:
   \begin{align*}
     r ( a b) = (r a) b = a(rb) \ , \quad r \in R, a,b \in A
   \end{align*}
\end{definition*}

In the following let \(W\) be the Weyl group of a \(BN\)-pair \(B,N \subset G\) with generating set \(S \subset W\) which only contains elements
of order \(2\). Furthermore let \(\mathscr{H}\) be the \(R:=\Z[v,v^{-1}]\)-algebra with basis \(W=\{T_w | w \in W\}\) and multiplication given by
\begin{align*}
    T_s T_w = \begin{cases}
        T_{sw}, & l(sw)=l(w)+1 \\
        v^2 T_{sw} + (v^2-1) T_w, & l(sw)= l(w)-1 \\
    \end{cases}
\end{align*}

\begin{exercise}
    Let \(\mathscr{A}\) be an \(R\)-algebra and \(\{A_s | s \in S\}\) such that
    \begin{enumerate}[label=\arabic*)]
        \item \(A_s^2 = v^2 1_\mathscr{A} + (v^2-1) A_s\)
        \label{first}
        \item \(A_s A_t A_s \dots = A_t A_s A_t \dots\), where both terms have \(m_{st}:=ord(st)\) factors and \(s \neq t\)
        \label{(M)}
    \end{enumerate}
    Show the following:
    \begin{enumerate}
        \item For every \(w \in W\) there is a well-defined element \(A_w \in \mathscr{A}\) such that 
            \begin{align*}
                A_w = A_{s_1} \dots A_{s_m}
            \end{align*}
        for any reduced expression \(w = s_1 \dots s_n, \ s_i \in S\). 
        \item There is a unique algebra homomorphism \(\varphi: \mathscr{H} \rightarrow \mathscr{A}\) such that \(\varphi(T_s)= A_s, s \in S\). 
        In particular, we have
        \begin{align*}
            \varphi(T_w) = A_w, \ w \in W
        \end{align*}
        \item There is a unique algebra homomorphism \(\varepsilon: \mathscr{H} \rightarrow R\) such that \(\varepsilon(T_s)= -1, s \in S\). 
        In particular, we have
        \begin{align*}
            \varepsilon(T_w)= (-1)^{l(w)}, \ w \in W
        \end{align*}
    \end{enumerate}
\end{exercise}

\begin{proof}
    \begin{enumerate}
        \item Since the multiplication on \(\mathscr{A}\) is associative and we have a neutral element \(\mathcal{M}:=(\mathscr{A},\cdot, 1)\) is a monoid.
        We can understand the family \((A_s)_{s \in S}\) as a map \(a: S \rightarrow \mathcal{M}\). Now to use Matsumoto's theorem
        we need that for any two \(s,t \in S\) distinct we have
        \begin{align*}
            A_s A_t A_s \dots = A_t A_s A_t \dots , 
        \end{align*}
        where both terms have \(m_{st}:=ord(st)\) factors. But this is given by \ref{(M)} thus we directly get a map
        \begin{align}
            \hat{A}: W \rightarrow \mathscr{A}, \hat{A}(w)=A_{s_1}\dots A_{s_m} 
            \label{Matsu}
        \end{align}
        for any reduced expression \(w = s_1 \dots s_n, \ s_i \in S\). 
        \item We have the basis \(\{T_w | w \in W\}\) of \(\mathscr{H}\) thus we can define an \(R\)-linear map
        \begin{align*}
            \varphi : \mathscr{H} \rightarrow \mathscr{A}, \ T_w \mapsto A_w
        \end{align*}
        If we can show that
        \begin{align*}
            \varphi(T_s T_w)=\varphi(T_s)\varphi(T_w) \tag{\(*\)}
            \label{*}
        \end{align*}
        for \(s \in S, w \in W\) then we showed that \(\varphi\) is indeed an algebra homomorphism. 
        Like we have learned in the lecture we make an induction over the length of \(w\): \\
        \underline{\(l(w)=0\):} Then we have \(w=1\) and \(T_1 = 1 \in \mathscr{H}\). By \eqref{Matsu} we know that \(A_1 = 1 \in \mathscr{A}\). 
        Thus \eqref{*} is clearly fulfilled. \\
        \underline{\(l(w)>0\):} \\
        Suppose the length goes up \(l(sw)=l(w)+1\), then we have
        \begin{align*}
            \varphi(T_s T_w) = \varphi(T_{sw})= A_{sw} \overset{\eqref{Matsu}}{=} A_s A_w = \varphi(T_s) \varphi(T_w)
        \end{align*}
        Suppose the length goes down \(l(sw)=l(w)-1\), then we have \(T_s T_w = v^2 T_{sw} + (v^2 -1) T_w\) and since \(\varphi\)
        is an \(R\)-linear
        \begin{align*}
            \varphi(T_s T_w) = v^2 \varphi(T_{sw}) + (v^2 -1) \varphi(T_w)
        \end{align*}
        Now since for \(s(sw)\) the length goes up, i.e. \(l(s(sw))= l(w)= l(sw)+1\) we know from above that
        \begin{align*}
            \varphi(T_w)=\varphi(T_{s(sw)})=\varphi(T_s) \varphi(T_{sw})
        \end{align*}
        Thus we can continue the penultimate equation with
        \begin{align*}
            &= (v^2 1 + (v^2-1)A_s)\varphi(T_{sw}) \\
            &\overset{\ref{first}}{=} A_s^2 \varphi(T_{sw}) \\
            &= \varphi()
        \end{align*}

    \end{enumerate}
\end{proof}

\end{document}