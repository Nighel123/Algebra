\documentclass[]{article}


\usepackage{imports}
\usepackage{enumitem}

\title{Algebraic Groups and Representations}
\author{Prof. Geck}

\begin{document}
\maketitle


Let \(K\) be an algebraically closed field and \(A:=K[T_1, \dots, T_n]\) the \(K\)-Algebra of the polynomial ring
in \(n\) indeterminantes. Each \(f \in A\) defines a map
\begin{align*}
    \dot{f}:K^n \rightarrow K, (x_1, \dots, x_n) \mapsto f(x_1, \dots, x_n)
\end{align*}

\begin{definition}
    We call a subset \(V \subseteq K^n\) \textbf{algebraic} iff there exists a set \(S \subseteq A\) such that
    \begin{align*}
        V=\{x \in K^n \ \vert \ \forall f \in S: f(x)=0\} = \bigcap_{f \in S} f^{-1}\{0\}
    \end{align*}
\end{definition}

\begin{remark*}
    Let \(V \subseteq K^n\) be algebraic then 
there is a set \(S \subseteq A\) such that
            \begin{align*}
                f |_V = 0, \ \forall f \in S
            \end{align*}

\end{remark*}

\begin{theorem*}
    The complements of algebraic sets \(V \subseteq K^n\) form a topology called the \textbf{Zariski topology}
\end{theorem*}

\begin{proof}
    \begin{itemize}
        \item \(V=\emptyset\) is algebraic because for \(S=\{1\}\) we have \(f^{-1}\{0\} = \emptyset\) for \(f \in S\)
        \item \(V=K^n\) is algebraic because for \(S= \emptyset\) we have 
        \begin{align*}
            \bigcap_{f \in S} f^{-1}\{0\} = K^n
        \end{align*}
        \item \(V_i\) algebraic, \(i \in I \ \Rightarrow V:= \bigcap_{i \in I} V_i\) is algebraic: \\
        Let \(S_i \subseteq A\) such that \(V_i = \left\{ x \in K^n | \forall f \in S_i: f(x)=0 \right\}\). Set
        \begin{align*}
            S:= \bigcup_{i \in I} S_i
        \end{align*}
        claim: \(V = \{x \in K^n | \forall f \in S: f(x)=0\}\)
        \begin{itemize}
            \item For \(x \in V\) we know that \(x \in V_i, \forall i \in I\). Now let \(f \in S\), then \(f \in S_i\) for some \(i \in I\).
            But since \(x \in V_i\) we know that \(f(x) = 0\)
            \item Now let \(x \in K^n\) with \(f(x) = 0, \forall f \in S\). Then for any \(i \in I\) we know that \(f(x) = 0, \forall f \in S_i\).
            But then \(x \in V_i\). But this implies \(x \in V\).
        \end{itemize}
        \item Suppose \(V_1, \dots , V_n\) are algebraic and \(S_i \subseteq A, i = 1, \dots, n\) such that 
        \begin{align*}
            V_i = \bigcap_{f \in S_i} f^{-1}\{0\}
        \end{align*}
        Then set \(S = \bigcap_{i = 1}^n S_i\) and \(V = \bigcup_{i = 1}^n V_i\)
        \begin{itemize}
            \item For \(x \in V\) we have \(x \in V_i\) for one \(i = 1, \dots , n\). But since \(S \subseteq S_i\) we have \(f(x) = 0, \forall f \in S\)
            \item For \(x \in K^n\) with \(f(x) = 0, \forall f \in S\) we have 
        \end{itemize}
    \end{itemize}
    ???
\end{proof}





\section{Affine varieties and the Zariski topology}

\setcounter{theorem}{2}
\begin{definition}
    Let \(A\) be a commutative \(K\)-Algebra and \(a_1, \dots , a_d \in A \). Then define 
    \begin{align*}
        K[a_1, \dots , a_d] := \bigcap_{\substack{U \subset A \ \text{subalgebra} \\ a_1, \dots , a_d \in U}} U
    \end{align*}
    the \textbf{the subalgebra generated by} \(a_1, \dots , a_d\). This is the smallest subalgebra of A
    containing \(a_1, \dots , a_d\). On the other hand a \(K\)-algebra \(A\) is called \textbf{finitly generated} iff 
    there exists \(a_1, \dots , a_d \in A\) such that \(A = K[a_1, \dots, a_d]\). 
    \label{def:finitly generated}
\end{definition}

\begin{remark*}
    Let \(A\) be a commutative \(K\)-algebra, then the following statements are equivalent:
    \begin{enumerate}[label=\arabic*)]
        \item \(A\) is finitly generated
        \item There exists a 
        surjective algebra homomorphism \(K[T_1, \dots, T_n] \rightarrow A\).
        \item \(A = \{\sum_{i=1}^{n}k_i a_1^{\alpha_1^i} \cdot \dots \cdot a_d^{\alpha_d^i}\vert k_i \in K, \ \alpha_j^i, n \geq 0\}\)
    \end{enumerate} 
    In particular we have
    \begin{align*}
        A = K[a_1, \dots, a_d]= \{\sum_{i=1}^{n}k_i a_1^{\alpha_1^i} \cdot \dots \cdot a_d^{\alpha_d^i}\vert k_i \in K, \ \alpha_j^i, n \geq 0\}
    \end{align*}
\end{remark*}

\begin{proof}
    a) Let \(A\) be finitly generated by \(S = \{a_1, \dots , a_d\}\), then we have \(A = \bigcap_{\substack{U \subset A \ \text{subalgebra} \\ S \subset U}} U\). You can show that 
    \begin{align*}
        U = \{\sum_{i=1}^{n}k_i a_1^{\alpha_1^i} \cdot \dots \cdot a_d^{\alpha_d^i}\vert k_i \in K, \ \alpha_j^i, n \geq 0\} \leq A
    \end{align*}
    is a subalgebra of A. But this implies that \(U = A\). \\
    b) On the other hand let \(f: K[T_1, \dots, T_d] \rightarrow A\) be a
    surjective algebra homomorphism. Set \(a_1 := f(T_1), \dots, a_d := f(T_d)\), then for any \(p = \sum_{i=1}^{n} k_i X_1^{\alpha_1^i} \cdot \dots \cdot X_d^{\alpha^i_d} \in K[T_1, \dots, T_d], \alpha_j^i \in \mathbb{N}_0\) we have
    \begin{align*}
        f(p)=\sum_{i=1}^{n}k_i a_1^{\alpha_1^i} \cdot \dots \cdot a_d^{\alpha^i_d}    
    \end{align*}
    But then since \(f\) is surjective we have 
    \begin{align*}
        A = \{\sum_{i=1}^{n}k_i a_1^{\alpha_1^i} \cdot \dots \cdot a_d^{\alpha_d^i}\vert k_i \in K, \ \alpha_j^i, n \geq 0\}
    \end{align*}
\end{proof}

\begin{definition*}
   Let \(X \neq \emptyset\) and \(x \in X\), then the \textbf{evalutation map at} \(x\) is given by:
   \begin{align*}
    \varepsilon_x: Map(X,A) \rightarrow K, f \mapsto f(x)
   \end{align*}
\end{definition*}

\begin{definition}
    Let \(X \neq \emptyset\) and \(A \subseteq Map(X,K)\) a K-subalgebra. We call the pair \(A_X := (X, A)\) an \textbf{affine variety over} \(K\) iff
\begin{enumerate}[label=\roman*)]
    \item \(A\) is finitly generated and \(1 \in A\)
    \item For any \(x, y \in X\) with \(x \neq y\) there exists a map \(f \in A\) such that \(f(x) \neq f(y)\)  \\
    (i.e. \(A\) seperates the points of \(X\))
    \item For any algebra homomorphism \(\Phi: A \rightarrow K\) there exists \(x \in X\) such that \(\Phi = \varepsilon_x \) is the evalutation homomorphism
    (i.e. \(\Phi(f)= f(x), \ \forall f \in A\)) in other words:
    \begin{align*}
        \forall \Phi \in Hom_{Alg}(A,K): \exists x \in X: \Phi = \varepsilon_x
    \end{align*}
\end{enumerate}
\label{def:affine variety}
\end{definition}

\begin{definition*}[Ideal]
    Let \(A\) be a commutative \(K\)-Algebra and \(S \subseteq A\) then 
    \begin{enumerate}[label=\alph*)]
        \item a \textbf{two sided Ideal} \(I \subseteq A \) is denoted by \( I \trianglelefteq A\)
        \item  the \textbf{ideal generated by} \(S\) \textbf{in} \(A\) is given by
        \begin{align*}
            \langle S\rangle_A := \bigcap_{S \subseteq I \trianglelefteq A} I
        \end{align*}
    \end{enumerate}
\end{definition*}

\setcounter{theorem}{6}
\begin{definition}
    Let \(A_X\) be an affine variety. 
    \begin{enumerate}[label=\alph*)]
        \item For any \(S \subseteq A\) define the \textbf{vanishing set over} \(S\) by
        \begin{align*}
            \mathcal{V}_X(S) := \{x \in X \vert \ \forall f \in S: f(x)=0\}
        \end{align*}
        \item For any subset \(Y \subseteq X\) define the \textbf{vanishing ideal over} \(Y\) by 
        \begin{align*}
            \mathcal{I}_A(Y):= \{f \in A \vert \ \forall y \in Y: f(y) = 0\}
        \end{align*}
    \end{enumerate}
\end{definition}

\begin{example}
    Suppose \(A_X\) is an affine variety. 
    \begin{enumerate}
        \item For the constant function \(1: X \rightarrow K\) we have \(1 \in A\) by definition of an affine variety. Furthermore:
        \begin{align*}
            \mathcal{V}_X(\{1\}):= \{x \in X| 1(x) = 1= 0\} = \emptyset
        \end{align*}
        \item Since \(A \leq Map(X,K)\) is a \(K\)-subalgebra, it is in particular a \(K\)-subvectorspace and thus includes the constant
        \(0: X \rightarrow K\) fuction. Thus we can form the vanishing set 
        \begin{align*}
            \mathcal{V}_X(\{0\}) := \{ x \in X | 0(x)=0\} = X
        \end{align*}
        \item For each \( x \in X\) we have 
        \begin{align*}
            \mathcal{V}_X\{\text{ker} \ \varepsilon_x\} = \{x\}
        \end{align*}
        \begin{proof}
            For any \(x \in X , \ \varepsilon_x: A \subseteq Map(X, K) \rightarrow K \) is a \(K\)-algebra homomorphism und 
            thus \(\text{ker} \ \varepsilon_x \trianglelefteq A\) is an ideal. We have \(x \in \mathcal{V}_X(\text{ker} \ \varepsilon_x)\)
        \end{proof}
    \end{enumerate}
\end{example}

\setcounter{theorem}{10}
\begin{definition}
    Let \((X,A)\) be an affine variety, then calling
    \begin{align*}
        \{\mathcal{V}_X(S) | S \subseteq A\}
    \end{align*}
    closed sets induces a topology on \(X\) called the \textbf{Zariski topology}.
\end{definition}


\setcounter{section}{2}
\section{Morphisms of affine varieties}

\begin{definition}
    Let \(A_X=(X,A)\) and \(B_Y=(Y,B)\) affine varieties.
A map \(\varphi: X \rightarrow Y\) is called a \textbf{homomorphism of affine varities from} \(A_X\) to \(B_Y\)
        \begin{align*}
            \varphi: A_X \xrightarrow[\text{var}]{} B_Y
        \end{align*}
        iff
        \begin{align*}
            g \in B \Rightarrow g \circ \varphi \in A, \ \forall g \in B
        \end{align*}
        In a diagramm:
        \begin{center}
            \begin{tikzpicture}
                \node (X) {\(X\)};
                \node (Y) [right=of X] {\(Y\)} edge [<-] node[above] {\(\varphi\)} (X);
                \node (K) [below right=of X] {\(K\)} edge [<-] node[below left] {\(g \circ \varphi\)} (X) edge [<-] node[right] {\(g\)} (Y);
            \end{tikzpicture}
        \end{center}
        \label{def:homo-affine-variety}    
\end{definition}

\begin{remark*}
    Every homomorphism of affine varieties \(\varphi: A_X \xrightarrow[\text{var}]{} B_Y\) induces an \(K\)-algebra homomorphism
        \begin{align*}
            \varphi^*: &B \rightarrow A \\
                        &g \mapsto g \circ \varphi
        \end{align*}
        In a diagramm we that translates to:In a diagramm we that translates to: 
        \begin{center}
            \begin{align*}
                &\begin{tikzpicture}
                    \node (phi) {\(\varphi^*:B\)};
                    \node (A) [right=of phi] {\(A\)} edge [<-] (phi);
                    \node (Y) [below=1mm of phi] {\(Y\)};
                    \node (K) [below= of Y] {\(K\)} edge [<-] node[left] (g) {\(g\)} (Y);
                    \node (X) [below=1mm of A] {\(X\)};
                    \node (K) [below= of X] {\(K\)} edge [<-] node[right] (gphi) {\(g \circ \varphi\)} (X);
                    \draw[|->,  shorten >=2mm, shorten <=2mm] (g) -- (gphi);
                \end{tikzpicture}
                &\begin{tikzpicture}
                    \node (X) {\(X\)};
                    \node (Y) [right=of X] {\(Y\)} edge [<-] node[above] {\(\varphi\)} (X);
                    \node (K) [below right=of X] {\(K\)} edge [<-] node[below left] {\(\varphi^*(g)\)} (X) edge [<-] node[right] {\(g\)} (Y);
                \end{tikzpicture}
            \end{align*}
        \end{center}
\end{remark*}

\setcounter{section}{11}
\section{Fields of definition}

In the following let \(K\) be an algebraically closed filed i.e. every non-constant polynomial decomposes into linear factors, and \(K_0 \subseteq K\) a subfield.
Furthermore regard \(\Map(X,K)\) as a \(K_0\)-algebra by restricting the the scalar multiplication to \(K_0\).  \\
\noindent\rule{\textwidth}{0.4pt} %horizontal line

\begin{note*}
    Remember by definiton \ref{def:finitly generated} we called an \(K_0\)-algebra \(A_0 \subseteq \Map(X,K)\)
    finitly generated iff there exists \(a_1, \dots, a_n \in A_0\) such that
    \begin{align*}
        A_0=K_0[a_1, \dots, a_n]
    \end{align*}
\end{note*}

Similarly too definition \ref{def:affine variety} we define:
\setcounter{theorem}{4}
\begin{definition}
    Let \(X \neq \emptyset\) and \(A_0 \subseteq Map(X,K)\) be a \(K_0\)-subalgebra. The pair \((X,A_0)\) is called \textbf{affine \(K_0\)-variety} iff:
        \begin{enumerate}
            \item \(A_0\) is finitly generated and the constant map \(1 \in A_0\)
            \item \(A_0\) seperates the points in \(X\)
            \item Any \(K_0\)-algebra homo \(\Phi: A_0 \rightarrow K\) is the evalutation map \(\varepsilon_x = \Phi\) for some \(x \in X\)
        \end{enumerate}
    We call 
    \begin{align*}
        X(K_0):= \{
            x \in X \ | \ \forall f \in A_0: f(x)\in K_0
        \}
    \end{align*}
    the set of \(K_0\)\textbf{-rational points}.
\end{definition}

Analogous to \ref{def:homo-affine-variety} we define:

\begin{definition}
    Let \((X,A_0)\) and \((Y,B_0)\) be affiine \(K_0\)-varieties.
    \begin{enumerate}
        \item We call
        \begin{align*}
            \varphi: (X,A_0) \rightarrow (Y,B_0) \quad \text{(varieties)}
        \end{align*}
        a \textbf{homomorphism of \(K_0\)-varieties from \((X,A_0)\) to \((Y,B_0)\)} or short \textbf{\(K_0\)-morphism} iff
        \(\varphi:X \rightarrow Y\) and
        \begin{align*}
            g \in B_0 \Longrightarrow g \circ \varphi \in A_0, \quad \forall g \in B_0
        \end{align*}
        \item Analogous to the remark to \ref{def:homo-affine-variety} this induces a map
        \begin{align*}
            \varphi^*: B_0 &\rightarrow A_0 \\
                        g &\mapsto \varphi^*(g)=g \circ \varphi
        \end{align*}
    \end{enumerate}

\end{definition}

\setcounter{section}{12}
\section{Affine \(K_0\)-varieties}

\begin{definition*}[see V.2.1 in \cite{jan}] Let \(K\) be a field and \(L \supseteq K \supseteq S\).
    \begin{enumerate}
        \item We call \(S\) a \textbf{subfield of} \(K\) iff \(S\) is a subring such that for every element not equal to \(0\) it also contains its inverse
        \item We call \(L\) a \textbf{field extension of} \(K\) iff \(L\) is itself a field. 
        \item We call \(K\) \textbf{algebraically closed} iff every polynomial \(f \in K[X]\) decomposes into linear factors.
        \item Let \(f \in K[X]\) and \(\alpha_1, \dots, \alpha_n \in K\) are all roots of \(f\) in \(K\) then we know from analysis that we have
        \begin{align*}
            f=g \prod_{k=1}^{n}(X-a_k)^{m_k}
        \end{align*}
        for a \(g \in K[X]\). Then \(m_k \in \N^\times\) are uniquely determined and \(m_k\) is called \textbf{muliplicity of the root} \(\alpha_k\).
        \item Let \(L\) be a field extension of \(K\) 
        \begin{enumerate}
            \item Let \(\tilde{L} \supseteq K\) be another field extension of \(K\) then we call a map
                \(\varphi:L \rightarrow \tilde{L}\) a \textbf{\(K\)-homomorphism} if \(\varphi\) is a ring homomorphism such that 
                \begin{align*}
                    \varphi |_K = id_K
                \end{align*}
                if \(\varphi:L \rightarrow L\) and is bijective we call it a \textbf{\(K\)-automorphism}
            \item For \(\alpha \in L\) we define the field extension 
            \begin{align*}
                K(\alpha):= \bigcap_{\substack{L \supseteq K \\ \alpha \in L}} L \subseteq L
            \end{align*}
            \textbf{generated by \(K\) adding \(\alpha\)}.
            \item We call an element \(\alpha \in L\) \textbf{algebraic over} \(K\) iff
            there exists a \(f \in K[X]\) such that
            \begin{align*}
                f(\alpha) = 0
            \end{align*}
            \item We call \(L\) \textbf{algebraic} iff every \(\alpha \in L\) is algebraic.
            \item We call \(L\) an \textbf{algebraic closure} iff \(L\) is algebraic and and algebraically closed.
        \end{enumerate} 
    \end{enumerate}
\end{definition*}

Furthermore we know:

\begin{theorem*}[V.2.2 in \cite{jan}]
    Let \(L\) be a field extension of \(K\) and \(\alpha \in L\). Then the following statements are equivalent:
    \begin{enumerate}
        \item The element \(a \in L\) is algebraic over \(K\)
        \item We have \(K[\alpha]=K(\alpha)\)
        \item We have \([K(\alpha):K]:=\dim_K(K(\alpha)) < \infty\)
    \end{enumerate}
    If these statements are true, then a \(K\)-basis of \(K(\alpha)\) is given by 
    \begin{align*}
        \mathscr{B}:=(\alpha^0, \dots, \alpha^{n-1})
    \end{align*}
    for some \(n \in \N\).
\end{theorem*}

\begin{theorem*}
    Let \(K\) be a field \(\alpha \in K\) and \(f \in K[X]\) with \(f(\alpha)=0\) then the following statements are equivalent:
    \begin{enumerate}
        \item \(\alpha\) is a root with muliplicity \(1\)
        \item \(f'\) does not have a root \(\alpha\)
    \end{enumerate}
\end{theorem*}

Let \(K_0:= \F_p, \ p:\text{prime}\). 
From V.3.5 in \cite{jan} we know that there exists an algebraic closure \(K:=\overline{K_0}=\overline{\F_p}\). 
Thus every \(\alpha \in K\) is algebraic. But this implies by the above theorem that \([K_0(\alpha):K_0]< \infty\)
and that we have a \(K_0\)-basis \(\mathscr{B}:=(\alpha^0, \dots, \alpha^{n-1})\) of \(K_0[\alpha]\). Thus we have
\begin{align*}
    K_0(\alpha)=\{\sum_{k=0}^{n}\lambda_k \alpha^k \ | \ \lambda_k \in K_0, n \geq 0\}
\end{align*}

But since \(|K_0|=p\) we get \(|K_0(\alpha)|=p^n\). But from linear algebra we know that this implies 
\(K_0(\alpha) \simeq  \F_{p^n}\) \\

\underline{In the following let \(q:= p^n\) for some \(n \in \N^\times\) }

\begin{theorem*}
    then 
    \begin{align*}
        F_q:=\{\alpha \in K \ | \ \alpha^q = \alpha\} \subset K
    \end{align*}
    is a subfield of \(K\).
\end{theorem*}
\begin{proof}
    Let \(a,b \in F_q\), then 
    \begin{itemize}
        \item \((ab)^q=a^qb^q=ab\)
        \item We know from analysis that for \(p:\text{prime}\) we have \(p | \binom{p}{k}\) for \(k=1, \dots, p-1\). Thus we get
        \begin{align*}
            (a+b)^p = \sum_{k=0}^{p}\binom{p}{k}a^k b^{p-k} = a^p + b^p
        \end{align*}
        But this also implies:
        \begin{align*}
            (a+b)^q=a^q+b^q = a+b
        \end{align*}
        \item we have \(0^q=0\) and \(1^q=1\)
        \item \((a^{-1})^q=\frac{1}{a^q}=\frac{1}{a}=a^{-1}\)
    \end{itemize}
\end{proof}

Obviously we have
\begin{align*}
    F_q = \roots(X^q-X)
\end{align*}
But since for \(f:=X^q-X\) we have \(f'=qX^{q-1}-1=q-1 \neq 0\) we know by the above theorem that it does not have roots with muliplicity \(>1\).
But since \(K\) was algebraically closed it does have all of its \(q\) roots. Thus it follows:
\begin{align*}
    |F_q|=|\roots(X^q-X)|=q
\end{align*}
But from linear algebra we know that this implies \(F_q \simeq  \F_q = \F_{p^k}\)
All in all we prooved
\begin{align*}
    K =\bigcup_{k \in \N} F_{p^k}
\end{align*}

\begin{theorem*}
\begin{enumerate}
    \item The map
    \begin{align*}
        \gamma: K &\rightarrow K \\
                \alpha &\mapsto \alpha^q
    \end{align*}
    is a \(F_q\)-automorphism.
    \item For any \(m \in \N^\times \) we have \(\F_q \subset \F_{q^m}\)
    \item For any \(n, n' \in \N^\times\) we have
    \begin{align}
        \F_{p^n} \cup \F_{p^{n'}} \subseteq \F_{p^{n n'}} 
        \label{eq:bodyrelations}
    \end{align}
\end{enumerate}
\end{theorem*}
\begin{proof}
\begin{enumerate}
    \item     We have seen in the proof above that \(\gamma_q\) is indeed a ring homomorphism with \(0 \mapsto 0\) and \(1 \mapsto 1\).
    Furthermore we have for \(\lambda \in F_q\) and \(\alpha \in K\)
    \begin{align*}
        \gamma_q(\lambda \alpha) = \underbrace{\gamma_q(\lambda)}_{=\lambda} \gamma_q(\alpha) = \lambda \gamma_q(\alpha)
    \end{align*}
    Thus \(\gamma_q\) is a \(F_q\)-homomorphism.
    Now since it is not the \(0\)-map it is injective. Now let \(\beta \in K\), then consider the polynomial \(p:=X^q-\beta \in K[X]\). 
    Since \(K\) was algebraically closed, we find \(\alpha \in K\) such that 
    \begin{align*}
        \gamma_q(\alpha)=\alpha^q= \beta
    \end{align*}
    \item For \(\alpha \in \F_q\) we have \(\alpha^{q^m}= \alpha\), but this already implies \(\alpha \in \F_{q^m}\).
    \item Lets call \(q':=p^{n'}\), then by 2. we know that 
    \begin{align*}
        &\F_q \subset \F_{q^m} = \F_{p^{mn}}, \ \forall m \in \N^\times &\F_{q'} \subset \F_{q'^{m'}}= \F_{p^{m'n'}}, \ \forall m' \in \N^\times
    \end{align*}
    Now if we select \(m=n'\) and \(m'=n\) we get 
    \begin{align*}
        \F_{p^n} \cup \F_{p^{n'}} \subseteq \F_{p^{n n'}}
    \end{align*}
\end{enumerate}
\end{proof}

\begin{lemma}
    Let \((X,A_0)\) an affine \(\F_q\)-variety and \(A:=\langle A_0 \rangle_K\), then \((X,A)\) is an affine variety over \(K\) 
    and there exists a unique endomorphism of affine varieties \(F_q:X\rightarrow X\) such that
    \begin{align*}
        f \circ F_q = f^q, \quad \forall f \in A_0
    \end{align*}
    and \(F_q^*:A \rightarrow A\) has the following properties:
    \begin{enumerate}
        \item \(F_q^*\) is injective and \(\im F^*_q = \{f^q \ | \ f \in A\}\)
        \item For each \(f \in A\) there exists some \(m \in \N\) such that \(F_q^*(f)^m=f^{q^m}\) (left: concetenation; right: multiplication)
        \item \(X(\F_q) = X^{F_q}\) where 
        \begin{align*}
            &X(\F_q):=\{x \in X \ | \ \forall f \in A_0: f(x) \in \F_q\}, \quad \text{is the set of \(\F_q\)-rational points} \\
            &X^{F_q}:=\{x \in X \ | \ F_q(x)=x\}, \quad \text{is the set of points fixed by \(F_q\)}
        \end{align*}
    \end{enumerate}
    \label{Lemma: F_q^*}
\end{lemma}

\begin{proof}
    Let \(x \in X\) then we define
    \begin{align*}
        \Phi_x:= \Phi := \gamma_q \circ \varepsilon_x : A_0 &\rightarrow K \\
            f &\mapsto \gamma_q(f(x))
    \end{align*}
    Since we know that the evalutation homomorphism \(\varepsilon:A_0 \rightarrow K, \ f \mapsto f(x)\) 
    and the \(F_q\)-automorphism \(\gamma_q:K \rightarrow K, \ \alpha \mapsto \alpha^q\) are both ring homomorphisms also
    \(\Phi_x\) is one. Since \(\gamma_q\) is \(F_q\)-automorphism we get for \(\lambda \in F_q, f \in A_0:\)
    \begin{align*}
        \Phi(\lambda f ) = \gamma_q(\lambda f(x)) = \lambda \gamma_q(f(x))= \lambda \Phi(f)
    \end{align*}
    Thus we get a \(F_q\)-algebra homo \(\Phi_x\) and since \((X,A_0)\) is an affine \(\F_q\)-variety we know that by 1.5 in \cite{kehr} that there
    is a unique \(x' \in X\) such that \(\Phi_x = \varepsilon_{x'}\). All together we have defined a map
    \begin{align*}
        F_q: X &\rightarrow X \\
            x &\mapsto x'
    \end{align*}
    Now let \(f \in A_0\) and \(x \in X\), then we have
    \begin{align*}
        (f \circ F_q)(x)=f(F_q(x))=f(x')=\varepsilon_{x'}(f)=\Phi_x(f)=\gamma_q(f(x))=(f(x))^q
    \end{align*}
    Thus we get \(f \circ F_q = f^q, \ \forall f \in A_0\). What makes \(F_q: (X, A_0) \rightarrow (X, A_0)\) a \(\F_q\)-endomorphism, since \(f^q \in A_0\).
    
    Uniqueness: Suppose we have another \(\F_q\)-endomorphism \(G:(X,A_0) \rightarrow (X,A_0)\) such that \(f \circ G = f^q\). 
    Suppose there is an \(x \in X\) such that \(G(x) \neq F_q(x)\), then since \(A_0\) seperates the points of \(X\) we would have
    an \(f \in A_0\) such that
    \begin{align*}
        f(G(x))\neq f(F_q(x))
    \end{align*}
    But this can't be true since
    \begin{align*}
        (f \circ G)(x)=f^q(x)=(f \circ F_q)(x)
    \end{align*}

    Now we want to proove that \(F_q:X \rightarrow X\) is a endomorphism of affine varieties: \\
    We have \(A = \langle A_0 \rangle_K\). Thus every \( f \in A \) can be written as a linear combination 
    \begin{align*}
        f = \sum_{i=1}^{n}\lambda_i f_i, \quad f_i \in A_0, \lambda_i \in K
    \end{align*}
    But then we have
    \begin{align*}
        (f \circ F_q)(x)= \sum_{i=1}^{n}\lambda_i f_i(F_q(x)) = \sum_{i=1}^{n}\lambda_i (f_i(x))^q= \sum_{i=1}^{n}\lambda_i f_i^q(x) 
    \end{align*}
    This shows 
    \begin{align*}
       f \circ F_q = \sum_{i=1}^{n}\lambda_i f_i^q \in \langle A_0 \rangle_K = A
    \end{align*}
    and we have prooven that \(F_q:X \rightarrow X\) is a morphism. Thus we obtain its induces algebra homomorphism
    \begin{align*}
        F_q^*:A &\rightarrow A \\
                f  &\mapsto F_q^*(f)= f \circ F_q
    \end{align*}
        
    We want to show that \(F_q^*\) is injective: \\
    
    For this we first proove that
    \begin{align*}
        \gamma_q^*:A &\rightarrow A \\
                    f &\mapsto f^q
    \end{align*}
    (where the potency is the multiplication of functions) is a \(\F_q\)-algebra homo:
    Suppose we have \(f,g \in A\) and \(\lambda \in \F_q\), then we have for \(x \in X\)
    \begin{align*}
        (f+g)^q(x) = ((f+g)(x))^q&=(f(x)+g(x))^q \\
        &=\sum_{k=0}^{q}\binom{q}{k}f(x)^k g(x)^{q-k} \\
        &=f(x)^q + g(x)^q \\
        &=(f^q+g^q)(x)
    \end{align*}
    Furthermore we certainly have \((fg)^q = f^q g^q\) and \(1^q=1, \ 0^q = 0\). And finaly \(\gamma_q^*\) is \(\F_q\)-linear, since 
    \begin{align*}
        \gamma_q^*(\lambda f)= \lambda^q f^q = \lambda f^q
    \end{align*}

    Now \(\gamma_q^*\) is also injective, since for \(f \in A \subseteq \Map(X,K)\)
    \begin{align*}
        \gamma_q^*(f)=f^q=0 \quad \Longrightarrow \quad f = 0
    \end{align*}

    Now we get go proove that \(F_q^*\) is injective: \\
    First we notice that for \(f \in A_0\) we have
    \begin{align*}
        F_q^*(f)= f \circ F_q = f^q = \gamma_q(f) = 0 \quad \Longrightarrow \quad f = 0
    \end{align*}
    Now let \(f \in \ker(F_q^*) \subseteq = A = \langle A_0 \rangle_K\) then select \(n \in \N\) minimal such that
    \begin{align*}
        f = \sum_{i=1}^{n} \lambda_i f_i, \quad \lambda_i \in K, f_i \in A_0
    \end{align*}
    Then \((f_1, \dots, f_n)\) are linearly independant, since if one of them was dependant on the others, then we would be able to
    represent \(f\) by a shroter sum. Next, consider the polynomials \(p_i:=X^q-\lambda_i \in K[X], \ i=1, \dots, n \). 
    Since \(K\) was algebraically closed we find \(\mu_i \in K\) such that \(\mu_i^q=\lambda_i\). But then we get:
    \begin{align*}
        0 = F_q^*(f)=\sum_{i=1}^{n} \lambda_i F_q^*(f_i) = \sum_{i=1}^{n} (\mu_i f_i)^q = \left( \sum_{i=1}^{n} \mu_i f_i \right)^q
    \end{align*}
    But this implies \(\sum_{i=1}^{n} \mu_i f_i = 0\) and thus \(\mu_i = \lambda_i = 0, \ i=1, \dots, n\) and thus \(f=0\).

    Now we want to proove that \(\im F_q^* = A^q:=\{f^q \ | \ f \in A\}\):\\
    Again let \(f = \sum_{i=1}^{n} \lambda_i f_i \in A, \ \lambda_i \in K, f_i \in A_0\) then we have
    \begin{align*}
        f^q = \gamma_q^*(f)&=\sum_{i=1}^{n} \gamma_q^*(\lambda_i f_i) \\
        &=\sum_{i=1}^{n} \lambda_i^q f_i^q \\
        &=\sum_{i=1}^{n} \lambda_i^q F_q^*(f_i) \\
        &=F_q^*\left(\sum_{i=1}^{n} \lambda_i^q f_i\right)
    \end{align*}
    And on the other hand select \(\mu_i^q=\lambda_i\) like above, then we get
    \begin{align*}
        F_q^*(f) = \sum_{i=1}^{n} (\mu_i f_i)^q = \left( \sum_{i=1}^{n} \mu_i f_i \right)^q \in A^q 
    \end{align*}
    Which prooves the claim. 

    Now we want to show that we have an \(m \in \N\) such that 
    \begin{align*}
        (F_q^*)^m(f) = f^{(q^m)}
    \end{align*}
    (on the left side the potency means concetenation and on the right side multiplication): \\
    Again let \(f = \sum_{i=1}^{n} \lambda_i f_i \in A, \ \lambda_i \in K, f_i \in A_0\) Then by \eqref{eq:bodyrelations} we have an \(m \in \N\) such that 
    \(\lambda_i \in \F_{p^m}, \ \forall i=1, \dots, n\). Thus we have \(\lambda_i^{g^m}=\lambda_i\) and we get:
    \begin{align*}
        F_q^*(f)^m = f \circ F^m_q &= \sum_{i=1}^{n} \lambda_i f_i \circ F^m_q \\
        &= \sum_{i=1}^{n} \lambda_i f_i^{g^m} \\
        &= \sum_{i=1}^{n} \lambda_i^{q^m} f_i^{q^m} \\
        &= \sum_{i=1}^{n} (\lambda_i f_i)^{q^m} \\
        &= \sum_{i=1}^{n} \gamma^m_q(\lambda_i f_i) \\
        &= \gamma^m_q\left(\sum_{i=1}^{n} \lambda_i f_i\right) \\
        &= \left(\sum_{i=1}^{n} \lambda_i f_i\right)^{q^m} \\
        &= f^{q^m} \\
    \end{align*}
    Now we want to proove 3.: \\
    Let \(x \in X(\F_q)\), then we have \(f(x) \in \F_q, \ \forall f \in A_0\) and thus we get
    \begin{align*}
        f(x)=f(x)^q=(f \circ F_q)(x)=f(F_q(x)), \ \forall f \in A_0
    \end{align*}
    But since \(A_0\) seperates the points of \(X\) this implies \(x = F_q(x)\) and thus \(x \in X^{F_q}\). This argument also works the other way around.
\end{proof}

\setcounter{theorem}{3}
\begin{remark}
    In the set-up of Lemma \ref{Lemma: F_q^*} \(A_0\) is uniquely determined by \(F_q\) with
    \begin{align*}
        A_0= \{f \in A \ | \ F^*_q(f)= f^q\}
    \end{align*}
\end{remark}
\begin{proof}
    We already know that for any \(f \in A_0\) we have \(F^*_q(f) = f \circ F_q = f^q\), thus "\(\subseteq\)" is already true. \\
    On the other hand let \(f \in A\) such that \(F^*_q(f)= f^q\). Since we have \(A:=\langle A_0 \rangle_K\) we can write
    \begin{align*}
        f = \sum_{i=1}^{n} \lambda_i f_i, \quad \lambda_i \in K, f_i \in A_0
    \end{align*}
    with \(n\in \N\) minimal such that \((f_i)_{i=1, \dots , n}\) are linearly independant. To be continued...
\end{proof}

\begin{definition}
    \label{Def:Frobenius map}
    Let \((X,A)\) be an affine variety over \(K\). A morphism \(F:X \rightarrow X\) is called \textbf{Frobenius with respect to} \(\F_q\) iff
    \begin{enumerate}
        \item the induced \(K\)-algebra homomorphism \(F^*:A \rightarrow A, f \mapsto f \circ F\) is injective
        \item \(\im(F^*)=A^q=\{f^q\ | \ f \in A\}\)
        \item for each \(f \in A\) there exists an \(m \in \N^\times\) such that
        \begin{align*}
            (F^*)^m(f)=f^{q^m}
        \end{align*}
    \end{enumerate}
    If we set
    \begin{align*}
        A_0:=\{ f \in A \ | \ F^*(f)=f^q\} \subseteq A
    \end{align*}
    then you can check i.t.h that this is a \(\F_q \)-subalgebra of \(A\), since we have
    \begin{align*}
        F^*(\lambda f)=\lambda F^*(f)=\lambda f^q = \lambda^q f^q = (\lambda f)^q, \quad \lambda \in \F_q, f \in A_0
    \end{align*}
\end{definition}

\begin{definition/theorem}
    Let \((X,A)\) an affine variety over \(K\) and \(F:X \rightarrow X\) Frobenius with respect to \(\F_q\), then we call the map
    \begin{align*}
        \sigma: A &\rightarrow A \\
        f &\mapsto \sigma(f):=(F^*)^{-1}(f^q)
    \end{align*}
    \textbf{arithmetic Frobenius map}. This map is well-defined and we have
    \begin{enumerate}
        \item \(F^* \circ \sigma = \sigma \circ F^*\)
        \item \(\sigma\) is an \(\F_q\)-algebra automorphism, i.e. a \(\F_q\)-linear ring isomorphism
        \item For every \(f \in A\) there exists some \(m \in \N^\times\) such that \(\sigma^m(f)=f\)
        \item For \(\lambda \in K\) and \(f \in A\) we have 
        \begin{align*}
            \sigma(\lambda f) = \lambda^q \sigma(f)
        \end{align*}
    \end{enumerate}
    \label{theorem: arithmetic-Frobenius}
\end{definition/theorem}
\begin{proof}
    Well-defined: \\
    By propery 1. \& 2. in \ref{Def:Frobenius map} we know that there exists an inverse map
    \begin{align*}
        (F^*)^{-1}:A^q \rightarrow A
    \end{align*}
    (which is also an \(K\)-algebra-homomorphism, in particular \(A^q\) is a \(K\)-algebra).
    \begin{enumerate}
        \item Let \(f \in A\), then
        \begin{align*}
            (\sigma \circ F^*)(f)=\sigma(F^*(f))=(F^*)^{-1}((F^*(f))^q)=(F^*)^{-1}(F^*(f^q))=f^q
        \end{align*}
        and on the other hand
        \begin{align}
            (F^* \circ \sigma)(f)=F^*(\sigma(f))=F^*((F^*)^{-1}(f^q))=f^q
            \label{eq:Fstar_sigma}       
        \end{align}
        This shows \(\sigma \circ F^* = F^* \circ \sigma\).
        \item We want to show that \(\sigma\) is \(\F_q\)-algebra automorphism: \\
        Now remember the map \(\gamma^*_q:A \rightarrow A, f \mapsto f^q\) that we used in the proof of \ref{Lemma: F_q^*}.
        We found there that this an injective \(\F_q\)-algebra homomorphism. But since we have
        \begin{align*}
            \sigma = (F^*)^{-1} \circ \gamma^*_q
        \end{align*}
        But since also 
        \begin{align*}
        (F^*)^{-1}:A^q \rightarrow A
        \end{align*}
        is an injective \(K\)-algebra homomorphism we get that \(\sigma\) is an injective \(\F_q\)-algebra homomorphism.
        
        What is left to show is that \(\sigma\) is surjective: \\
        Let \(f \in A\). Since by the assumptions \(F\) was Frobenius with respect to \(\F_q\), 
        we know that there exists \(m \in \N\) such that
        \begin{align*}
            (F^*)^m(f)=f^{q^m}
        \end{align*}
        Now since we have seen in \eqref{eq:Fstar_sigma} that \((F^* \circ \sigma)(f)=f^q\) and by 1. \(F^*\) and \(\sigma\) are commutative 
        we also get
        \begin{align*}
            (F^*)^m(\sigma^m(f))= ((F^*)^m \circ \sigma^m)(f) = (F^* \circ \sigma)^m(f) = f^{q^m}
        \end{align*}
        But from the injectivity of \(F^*\) follows also the injectivity of \((F^*)^m\) and thus we have
        \begin{align*}
            \sigma^m(f)=f
        \end{align*}
        This yields 3. and we also have \(\sigma(\sigma^{m-1}(f))=f\) what shows that \(\sigma\) is surjective.
        \item done (see above)
        \item Since \(F^*:A \rightarrow A\) is an \(K\)-algebra homo by \ref{Def:Frobenius map} also it's inverse has this property and we get:
        \begin{align*}
            \sigma(\lambda f) = (F^*)^{-1}(\lambda f)^q = \lambda^q (F^*)^{-1}(f^q) = \lambda^q \sigma(f)
        \end{align*}
    \end{enumerate}
\end{proof}

\begin{lemma}
    Let \((X,A)\) be an affine variety over \(K\) and \(F:X \rightarrow X\) Frobenius with respect to \(\F_q\).
    Furthermore let \(\sigma: A \rightarrow A\) be the arithmetic Frobenius map.
    \begin{enumerate}
        \item For any \(f \in A\) the \(K\)-vs \(V:=\langle \sigma^j(f) \ | \ j \in \N_0 \rangle_K \leq A\) 
        is finitly generated by elements \(f_1, \dots, f_n\in A\) that are fixed by \(\sigma\), i.e. \(\sigma(f_i)=f_i, \ i=1, \dots, n\).
        \item We have \(A_0=A^\sigma\) where 
        \begin{align*}
            A_0=\{f \in A \ | \ F^*(f)= f^q\}
        \end{align*}
        denotes the \(\F_q\)-algebra from \ref{Def:Frobenius map}. And \(A^\sigma:=\{ f \in A \ | \ \sigma(f)=f\}\) is the set of functions fixed
        by \(\sigma\). Furthermore also \(A_0\) is finitly generated and we have \(A=\langle A_0 \rangle_K\)
        \item Every \(\F_q\)-basis of \(A_0\) is an \(K\)-basis of \(A\). TODO: Equivalence to the map?
    \end{enumerate}
\end{lemma}
\begin{proof}
    \begin{enumerate}
        \item By the preciding theorem we know that there exists an \(m \in \N\) such that \(\sigma^m(f)=f\). This implies
        \begin{align*}
            \{\sigma^j(f)\ | \ j \in \N_0\} = \{f, \sigma(f), \dots, \sigma^{m-1}(f)\}
        \end{align*}
        and therefore \(V\) is finitly generated and thus finite dimensonal over \(K\). Now we know that \(\F_{q^m} \supseteq \F_q\) and since
        \(\F_{q^m}\) is finite it certainly has finite dimension over \(\F_q\). But for any \(\F_q\)-basis \((\xi_0, \dots \xi_{d-1})\) \(\F_{q^m}\) would have
        \(|\F_q |^d\)-many elements thus \(d=m\). Now we define the functions
        \begin{align*}
            \tilde{f_i} := \sum_{j=0}^{m-1}\sigma^j(\xi_i f) = \sum_{j=0}^{m-1}\xi_i^{q^j} \sigma^j(f), \quad\forall i=0, \dots, m-1
        \end{align*}
        where in the latter equality we used that \(\sigma(\xi_i f) = \xi_i^q \sigma(f)\) by \ref{theorem: arithmetic-Frobenius}. TODO. 
    \end{enumerate}
\end{proof}


\setcounter{section}{13}
\section{Representations of Algebras}

In the following we assume that all rings \(R\) have an identity \(1 \in R\) and that homomorphisms map identities onto each other.

\begin{definition}
    Let \(A\) be a ring with the multiplication \(\cdot: A^2 \rightarrow A\). We call \(A\) a \(K\)\textbf{-Algebra} iff \(A\) is a \(K\)-vs 
    with the action \(.:K\times A \rightarrow A\) and both maps commute with one another:
    \begin{align*}
        \lambda . (a_1 \cdot a_2) = (\lambda . a_1)\cdot a_2 = a_1 \cdot (\lambda . a_2), \qquad \lambda \in K, a_i \in A, i=1,2
    \end{align*}
\end{definition}

\begin{definition*}[subalgebra]
    A \(K\)-subvs \(U \leq A\) is called \(K\)\textbf{-subalgebra of} \(A\) iff
    \(U\) is also a subring of \(A\) i.e. that it is closed to multiplication:
    \begin{align*}
         a \cdot b \in U, \qquad \forall a,b \in U
    \end{align*}
\end{definition*}

\begin{note*}
We dont require \(U \leq A\) as subalgebra to include the same \(1 \in A\)! (see below 14.12 for an example)
\end{note*}

\setcounter{theorem}{4}
\begin{definition}
    Let \(A\) be a Ring. We call a set \(M\) an \(A\)-\textbf{left module} 
    iff there is a map \(.: A \times M \rightarrow M\) (called \textbf{action of} \(M\)) such that 
    \begin{enumerate}[label=\roman*)]
    \item \(a.(m+n) = a. m + a.n\)
    \item \((a+b).m=a.m+b.m\)
    \item \((ab).m=a.(b.m)\)
    \item \(1_A . m = m\)
    \end{enumerate}
    \(\forall a,b \in A, \ m,n \in M\).
\end{definition}

\begin{theorem*}
    Let \(A\) be \(K\)-algebra and \(M\) an \(A\)-left module, then
    \begin{align*}
        (\lambda a). m = \lambda .(a.m) = a.(\lambda.m)
    \end{align*}
    holds \(\forall a \in A, \lambda \in K\) and \(m \in M\).
\end{theorem*}

\setcounter{theorem}{7}
\begin{definition}
    Let \(M,N\) be \(A\)-modules. 
    \begin{enumerate}
        \item \(Hom_A(M,N)\) denotes the set of all \(A\)-module homomorphisms \(f:M \rightarrow N\) and 
            \begin{align*}
                End_A(M):= Hom_A(M,M)
            \end{align*} 
            is the set of all \(A\)-module endomorphisms \(f:M \rightarrow M\) into itself.
        \item We call an \(A\)-module \(M\) \textbf{irrueducible} or \textbf{simple}, iff \(M \neq 0\) and
            \(M\) and \(0\) are the only submodules of \(M\). Furthermore
            \begin{align*}
                Irr(A) := \{[M] \ | \ M \text{ is an irreducible } A-\text{module}\}
            \end{align*}
            denotes the set of all irreducible \(A\)-modules grouped into classes via isomorphism. 
    \end{enumerate}
\end{definition}

\begin{remark}
    Suppose we have two \(A\)-modules \(M,N\) and a \(K\)-linear map \(f: M \rightarrow N\). Then we have: 
    \(f\) is an \(A\)-modul homomorphism iff
    \begin{align*}
        f \circ \Phi_M(a) = \Phi_N(a) \circ f, \qquad \forall a \in A
    \end{align*}
    where \(\Phi_M, \Phi_N\) are the corresponding representations of \(M\) and \(N\). In a diagram:
    \begin{center}
        \begin{tikzpicture}
            \node (M1) {\(M\)};
            \node (N1) [right=of M1] {\(N\)} edge [<-] node[above] {\(f\)} (M1);
            \node (M2) [below=of M1] {\(M\)} edge [<-] node[left] {\(\Phi_M\)} (M1);
            \node (N2) [right=of M2] {\(N\)} edge [<-] node[below] {\(f\)} (M2) edge [<-] node[right] {\(\Phi_N\)} (N1);
            \node at ($(M1)!0.5!(N2)$) {\(\circlearrowleft\)};
        \end{tikzpicture}
    \end{center}
\end{remark}
\begin{proof}
    Let \(a \in A, m \in M\). Then we have 
\begin{gather*}
    (f \circ \Phi_M(a))(m) = f(am) = a f(m) = ( \Phi_N \circ f )(m) \\
    \Longleftrightarrow \\
    f \text{ is an } A\text{-module homomorphism}
\end{gather*}
\end{proof}


\begin{remark}
    Let \(A\) be a finite dimensional \(K\)-Algebra (i.e. the underlying \(K\)-vs has finite dimension), then every irreducible
    \(A\)-module has finite dimension.
\end{remark}
\begin{proof}
    Let \(M\) be an irreducible \(A\)-module. Then sinc \(M \neq 0\) we have \(m \in M\setminus 0\). Now consider
    \begin{align*}
        \ell_m: A &\rightarrow M\\
            a&\mapsto a.m
    \end{align*}
    the \textbf{left multiplication of algebra elements with} \(m\). We know from last year that this is an \(A\)-module homomorphism.
    By the isomorphism theorem we get:
    \begin{align*}
        A / \ker \ell_m \simeq \im \ \ell_m =Am \leq M, \quad \text{as \(A\)-modules}
    \end{align*}
    Since \(M\) was irreducible we have \(Am = M\) and thus
    \begin{align*}
        \infty > \dim_K A \geq \dim_K A - \dim_K \ker \ell_m = \dim_K M
    \end{align*}
\end{proof}
\begin{definition*}
Let \(R,\cdot,+\) be a ring then the \textbf{opposite ring} \(R^{op}:=(R,*,+)\) is the same ring 
with the opposite multiplication \(*:R^2 \rightarrow R\)
\begin{align*}
    r * s := s \cdot r, \qquad r,s \in R
\end{align*}
\end{definition*}
\begin{theorem}
    Let \(A\) be a \(K\)-algebra and \(M\) be an \(A\)-module. For any \(A\)-module \(V\) we define an 
    \(\End_A(M)\)-module structure on \(\Hom_A(M,V)\) via:
    \begin{align*}
        \End_A(M)^{op} \times \Hom_A(M,V) &\rightarrow \Hom_A(M,V) \\
        ( \varphi, f) \mapsto \varphi.f := f \circ \varphi
    \end{align*}
\end{theorem}
\begin{proof}
    Let \(\varphi, \psi \in \End_A(M)^{op}\) and \(f,g \in \Hom_A(M,V)\). Then we get:
    \begin{enumerate}[label=\roman*)]
        \item \(\varphi.(f+g)=(f+g) \circ \varphi = f \circ \varphi + g \circ \varphi = \varphi .f+\varphi .g\)
        \item \((\varphi+\psi).f = f \circ (\varphi + \psi) = f \circ \varphi + f \circ \psi = \varphi.f+\psi.f\)
        \item \((\varphi * \psi).f=(\psi \circ \varphi).f = (f \circ \psi) \circ \varphi = \varphi.(\psi.f)\)
        \item \(\id_M.f=f \circ id_M = f\)
    \end{enumerate}
\end{proof}
\begin{example}
    Let \(A\) be a \(K\)-algebra and \(e \in A\) an idempotent. Then  
   \begin{enumerate}
    \item For any \(A\)-module \(M\) 
        \begin{align*}
            \Phi:\Hom_A(Ae,M) &\rightarrow eM \\
                            f &\mapsto ef(e)
        \end{align*}
    is an isomorphism of \(K\)-vectorspaces.
    \item In the case of \(M=Ae\) in 1.
        \begin{align*}
        \Phi:\End_A(Ae)^{op} \cong eAe
        \end{align*}
    is an isomorphism of \(K\)-algebras.
   \end{enumerate} 
\end{example}
\begin{proof}
    \begin{enumerate}
        \item We already know that \(Ae = \im \ell_e\) as the image of left multiplication is an \(A\)-algebra.
        
        \begin{itemize}
            \item[\underline{\(eM \leq M\) as \(K\)-vs:}]
                    \begin{itemize}
                        \item \(em+en=e(m+n) \in eM\)
                        \item \(kem = e(km) \in eM\)
                    \end{itemize} 

            \item[\underline{\(\lnot eM \leq M\) as \(A\)-mod:}]
            \(aem \neq eav\) in general 

            \item[\underline{\(\Phi\) is well defined:}] Since \(e = ee \in Ae\) the expression \(f(e)\) for \(f \in \Hom_A(Ae,M)\) is well defined.
            \item[\underline{\(\Phi\) is \(K\)-linear:}] Let \(f,g:Ae \rightarrow M\) be \(A\)-Mod hom., then we have
                \begin{align*}
                    \Phi(f+g)=e(f(e)+g(e))=ef(e)+eg(e) = \Phi(f) + \Phi(g)
                \end{align*}
            and additionally let \(\lambda \in K\), then
            \begin{align*}
                \Phi(\lambda f) = e(\lambda f)(e) = e (\lambda f(e)) = \lambda e f(e) = \lambda f(e)
            \end{align*}
            \item[\underline{\(\Phi\) is injective:}] Suppose \(f,g: Ae \rightarrow M\) are \(A\)-mod hom. with
            \begin{align*}
                \Phi(f)=ef(e)=eg(e)=\Phi(g)
            \end{align*}
            We want to show that \(f=g\). So let \(ae \in Ae\), then
            \begin{align*}
                f(ae)=af(e)=af(ee)=aef(e)=aeg(e)=ag(e)=g(ae)
            \end{align*}
            \item[\underline{\(\Phi\) is surjective:}] Suppose we have \(em \in eM\). We want to construct \(f:Ae \rightarrow M\) (\(A\)-mod) such that
            \begin{align*}
                \Phi(f)=ef(e)=em
            \end{align*}
            \underline{claim:} We have \(f:Ae \rightarrow M, ae \mapsto am\) \(A\)-mod:
            \begin{proof}
                \begin{itemize}
                    \item \(f(ae+be)=f((a+b)e)=(a+b)m=am+bm=f(ae)+f(be)\)
                    \item \(f(a(be))=f((ab)e)=(ab)m =a(bm)=af(be)\)
                \end{itemize}
            \end{proof}
            But this means
            \begin{align*}
                \Phi(f)=ef(e)=ef(ee)=e(em)=em
            \end{align*}
        \end{itemize}
        \item If we set \(M=Ae\) in 1., then \(\Phi\) turns into a \(K\)-vs iso.
            \begin{align*}
        \Phi:\End_A(Ae) &\rightarrow eAe \\
                        f &\mapsto ef(e)
        \end{align*}
        Now since we know that \(\End_A(Ae)\) is an \(K\)-algebra we also know that \(eAe\) is one. Now consider \(f,g \in \End_A(Ae)^{op}\), then we have
        \(f(e)=ae=aee=f(e)e\) for \(a \in A\).
        \begin{align*}
            \Phi(f * g)= e(g \circ f)(e) =e(g(f(e))) = e(g(f(e)e))=e(f(e)g(e))=(ef(e))(g(ee))=(ef(e))(eg(e))=\Phi(f)\Phi(g)
        \end{align*}
        Thus we get \(\Phi:\End_A(Ae)^{op} \cong eAe\) as \(K\)-algebras.
    \end{enumerate}
\end{proof}

\underline{In the following let \(G\) be a finite group, \(B \leq G\) a subgroup and \(A=\C[G]\) be the group algebra of \(G\) over \(\C\).}
\begin{theorem*}
    The element
    \begin{align}
        e = \frac{1}{|B|}\sum_{b \in B}^{} b \in \C[G]
        \label{eq:idempotent}
    \end{align}
    is an idempotent, i.e. \(e^2 = e\).
\end{theorem*}
\begin{proof}
    We have
    \begin{align*}
        e = \sum_{g \in G}^{}\lambda_g g, \quad \lambda_g :=
        \begin{cases}
            1/|B|, & g \in B \\
            0 & else
        \end{cases}
    \end{align*}
    And thus:
    \begin{align*}
        e^2 = \left(\sum_{g \in G}^{} \lambda_g g\right) ^2 &= \sum_{u\in G}^{}\left(\sum_{\substack{g,f \in G \\ fg=u}}^{} \lambda_f \lambda_g\right) u \\
        &=\sum_{u\in B}^{}\left(\sum_{\substack{a,b \in B \\ ab=u}}^{} \frac{1}{|B|^2}\right) u \\
        &=\frac{1}{|B|^2}\sum_{u\in B}^{}\left(\sum_{\substack{a,b \in B \\ ab=u}}^{} 1\right) u \tag{\(*\)}\\
    \end{align*}
    Now by I.6.1 in \cite{jan} we know that for any \(u,a \in B\) the map
    \begin{align*}
        \tau_a: B &\rightarrow B \\
                b &\mapsto ab
    \end{align*}
    is a bijection. Thus we get
    \begin{align*}
        \sum_{\substack{a,b \in B \\ ab=u}}^{} 1 = |B|
    \end{align*}
    But this means that \((*)\) completes to \(\frac{1}{|B|}\sum_{b \in B}^{} b = e\)
\end{proof}


\begin{theorem*}
    The map
\begin{align*}
    \cdot:(B \times B) \times G \rightarrow G
\end{align*}
defnies an operation of \(B \times B\) on \(X=G\) (see I.6.1 in \cite{jan}).
\end{theorem*}
\begin{proof}
    \begin{enumerate}
        \item We have
        \begin{align*}
            [(a,b)(c,d)]g=(ac,bd)g=acgd^{-1}b^{-1}=(a,b)[cgd^{-1}]=(a,b)[(c,d)g]
        \end{align*}
        \item and 
        \begin{align*}
            (1,1)g=1g1^{-1}=g
        \end{align*}
    \end{enumerate}
\end{proof}

Again from \cite{jan} we know that the orbits \(Orb(g)=BgB\) form a partition of \(G\) (I.6.1 in \cite{jan}). Thus we get

\begin{align}
    \bigsqcup_{\omega \in \Omega} B \omega B = G
    \label{eq:partition}
\end{align}
for a transversal \(\Omega \subseteq G\) (see I.1.14 in \cite{jan}) for an explanation.

\begin{proposition}
    A basis of \(e\C[G]e\) is given by \(e \Omega e\). In particular \(\dim_\C e\C[G]e = | \Omega |\). \\ Similarly we have
    \(\dim_\C \C[G]e = [G:B]\)
\end{proposition}
\begin{proof}
By the definition of \(\C[G]\) we have
\begin{align*}
    e \C[G]e &= \left\{ e \left(\sum_{g \in G}^{}\lambda_g g\right)e \ | \ \lambda_g \in \C\right\} \\
    &= \left\{ \left(\sum_{g \in G}^{}\lambda_g ege\right) \ | \ \lambda_g \in \C\right\} \\
    &= \text{span}_\C\{ege | g \in G\} =: \langle ege | g \in G \rangle_\C
\end{align*}
Now by \eqref{eq:partition} we have \(\omega \in \Omega\) for every \(g \in G\) such that \(g \in B \omega B\), i.e.
\(g = b \omega b'\) for some \(b,b' \in B\). Now by the definition of \(e\) \eqref{eq:idempotent} we have:
\begin{align*}
    eb = \frac{1}{|B|} \sum_{b'\in B}^{}b'b =e
\end{align*}
and also \(b'e=e\). All together: \\
For every \(g \in G\) we have a unique \(\omega \in \Omega\) such that 
\begin{align*}
    ege = eb\omega b'e=ewe
\end{align*}
What implies that \(e \Omega e\) generates \(e \C[G]e\). Now lets show linear independance of \(e \Omega e\): \\
Suppose we have \(\lambda_\omega \in \C, \omega \in \Omega\) such that
\begin{align*}
    0 = \sum_{\omega \in \Omega}^{}\lambda_\omega e \omega e 
\end{align*}
then again by the definition of \(e\) in \eqref{eq:idempotent} we have:
\begin{align*}
    \sum_{\omega \in \Omega}^{}\lambda_\omega e \omega e 
    &= \sum_{\omega \in \Omega}^{}\lambda_\omega \frac{1}{|B|^2} \sum_{b,b' \in B}^{}b \omega b' \\
    &= \sum_{\omega \in \Omega}^{}\lambda_\omega \frac{1}{|B|^2} \sum_{g \in B \omega B}^{} \underbrace{|\{(b,b')\in B^2| b \omega b' = g\}|}_{=n_g \in \N^\times} g \\
    &= \sum_{\omega \in \Omega}^{} \sum_{g \in B \omega B}^{} \frac{\lambda_\omega n_g}{|B|^2} g \\
\end{align*}
Now since \(G\) is a basis of \(\C[G]\) and \(\{B \omega B | \omega \in \Omega\}\) is a partition of \(G\), we get \(\frac{\lambda_\omega n_g}{|B|^2} = 0 \Rightarrow \lambda_\omega = 0\)

Now consider \(\C[G]e\): \\
Like above we have
\begin{align*}
    \C[G]e = \{ \sum_{g \in G}^{} \lambda_g g e \ | \ \lambda_g \in \C\} = \langle ge | g \in G \rangle_\C
\end{align*}

Again since the left cosets of g \(gB\) define a partition of \(G\) we can select a system of representatives \(T \subseteq G\).
Then for every \(g \in G\) we have a unique \(\xi \in T\) such that \(g \in \xi B\), i.e. \(g = \xi b, b \in B\). Again by the definition 
of \(e\) we get

\begin{align*}
    ge = \xi be = \frac{\xi}{|B|} \sum_{b' \in B}^{} b b' =\xi e
\end{align*}

Thus by the abouve we get \(\C[G]=\langle T e \rangle_\C\). Thus \(Te\) already is a generating system of \(\C[G]\). We still need to proove
that \(Te\) are linearly independant: \\
Suppose we have \(\lambda_\xi \in \C, \xi \in T\) such that 
\begin{align*}
    0 = \sum_{\xi \in T}^{} \lambda_\xi \xi e 
\end{align*}
Then again by the definition of \(e\) we get
\begin{align*}
    \sum_{\xi \in T}^{} \lambda_\xi \xi e &= \sum_{\xi \in T}^{} \sum_{b\in B}^{} \frac{\lambda_\xi}{|B|} \xi b \\
    &= \sum_{\xi \in T}^{} \frac{\lambda_\xi}{|B|} \sum_{b\in B}^{}  \xi b \\
    &= \sum_{\xi \in T}^{} \frac{\lambda_\xi}{|B|} \sum_{g \in \xi B}^{}  \underbrace{|\{b \in B \ | \ \xi b = g \}|}_{=:i_{\xi,g} \in \N^\times} g\\
    &= \sum_{\xi \in T}^{} \sum_{g \in \xi B}^{} \frac{\lambda_\xi}{|B|} i_{\xi,g} g\\
\end{align*}
Now since \(g \in G\) are linearly independant and \(TB\) is a partition of \(G\) we get that \(\frac{\lambda_\xi}{|B|} i_{\xi,g}=0, \forall \xi \in T, g \in G\).
But thia implies that \(\lambda_\xi = 0, \xi \in T\).

\end{proof}

\setcounter{section}{14}
\section{The structure of \(eAe\)}

\begin{definition*}
    Let \(K\) be a field and \(V\) be a \(K\)-vs. Then we call a map \(\beta : V \times V \rightarrow K\) \underline{bilinear form} if it is
    linear in both arguments. \\

    Let \(\beta:V ^2 \rightarrow K\) be a bilinear form. Then we make the following definitions:
    \begin{enumerate}
        \item If we have \(\beta(v,w) = \beta(w,v), \quad \forall v,w \in V\) then we call \(\beta\) \underline{symmetric}. 
        \item If \(\beta\) is a symmetric bilinear form:
            \begin{enumerate}
                \item We set          
                \begin{align*}
                S^\perp := \{ v \in V \ | \ \forall s \in S: \beta(v,s)=0\}
                \end{align*}
                \item the \underline{radical of \(\beta\)} is given by:
                \begin{align*}
                    \rad(\beta):= V^\perp = \{v \in V \ | \ \forall w \in V: \beta(v,w)=0\}
                \end{align*}
                and call \(\beta\) \underline{non-degenerate} if \(\rad(\beta)=0\)
            \end{enumerate}
    \end{enumerate}
\end{definition*}

\section{Schur relations}

In the following let \(K\) be a field, \(A\) a symmetric finite-dimensional \(K\)-algebra with associated trace form \(\tau:A \rightarrow K\) and 
\(\beta_\tau:A^2 \rightarrow K\) it's induced bilinear. Furthermore let \(\mathscr{A}=\{(a_1, \dots, a_l)\}\) be a basis of \(A\) and 
\(\mathscr{B}^\vee = \{a_1^\vee\}\)


\setcounter{section}{1}

\section{Groups with \(BN\)-pairs}
In the following let \(G\) a group with a \(BN\)-pair, i.e. \(H:= B \cap N \trianglelefteq N\) a normal subgroup and the Weyl group 
\(W=N/H=\langle S \rangle\) is generated by elements \(s \in S\) of order two \(s^2=1\). \\


Let \(M\) be a monoid (i.e. a set \(M\) with an associative multiplication \(m * m'\) and a neutral element \(1_M \in M\)). Furthermore
let \(f:S \rightarrow M\) be a map, such that for every pair \(s,t \in S, s \neq t\) with finite order \(m_{st}=ord(st) \in \N\) the products
\begin{align*}
    f(s)*f(t)*f(s)* \dots = f(t) * f(s) * f(t) *\dots \tag{M}
    \label{M}
\end{align*}
are equal where both of them have \(m_{st}\) factors. 
\setcounter{theorem}{7}

\begin{remark}
    Let \(s,t \in S, s \neq t\) such that \(m:=ord(st)< \infty\) is finite, then the products
    \begin{align*}
        sts \dots = tst \dots
    \end{align*}
    with both \(m\) factors are equal and are reduced expressions. 
    \label{remark:commutativity of products}
\end{remark}

\begin{theorem}[Matsumoto 1960's]
    Let \(M\) be a monoid and \(f:S \rightarrow M\) be a map satisfying \eqref{M}, then there is a unique map 
    \(\hat{f}:W \rightarrow M\) such that
    \begin{align*}
        \hat{f}(w)=f(s_1) * \dots * f(s_r)
    \end{align*}
    for every \(w \in W\) with reduced expression \(w=s_1 \dots s_r\). In particular \(\hat{f}\) is an extension of \(f\).
    \label{Matsumoto}
\end{theorem}

\begin{corollary}
    Under the assumptions of Theorem \ref{Matsumoto}, assume that \(M\) is a group such that \(f(s)^2=1, \ s \in S\), 
    then \(\hat{f}:W \rightarrow M\) is a group homomorphism. \label{corollary:Matsumoto}
\end{corollary}

\setcounter{section}{2}

\section{The Hecke algebra of a finite group with a \(BN\)-pair}
\setcounter{theorem}{6}
\begin{definition}
    Let \(I\) be a finite index set and \(M=(m_{ij})_{i,j \in I} \in \Z^{|I|}\) a matrix with ones on the diagonal, 
    i.e. \(m_{ii}=1,\ i \in I\) and otherwise the entries are \(\geq 2\) i.e. \(m_{ij} \geq 2, \ i, j \in I, i \neq j\)
    and symmetric \(m_{ij}=m_{ji}\). Then we call \(M\) a \underline{Coxeter Matrix}. Furthermore let 
    \(\{c_i\}_{i \in I} \subset \Z_{>0}\) be a family with \(c_i = c_j\) if \(m_{ij}\) is an odd integer. 
    Additionally let \(\mathscr{P}\) be an infinite set of prime powers (of potentially differeny primes) and 
    \begin{align*}
        \mathscr{F}=\{G(q) \ | \ q \in \mathscr{P}\}
    \end{align*}
    Then we say \(\mathscr{F}\) is a \underline{series of groups of type \(M\) with parameters \(\{c_i\}_{i \in I}\)}
    if it has the following properties:
    \begin{enumerate}
        \item For each \(q \in \mathscr{P}\), the group \(G(q)\) is finite and has a \(BN\)-pair 
        \begin{align*}
            B(q), N(q) \subseteq G(q)
        \end{align*}
        with Weyl group \(W(q)=\langle s_i^{(q)} \ | i \in I \rangle\) such 
        that the \((i,j)\)-entry of \(M\) is the order of \(s_i^{(q)} s_j^{(q)} \in W(q)\) i.e. \(m_{ij}=ord(s_i^{(q)}s_j^{(q)})\)
        \item For each \(q \in \mathscr{P}\) and \(i \in I\) we have 
        \begin{align*}
            |B(q) n_{s_i}^{(q)} B(q) | = q^{c_i} | B |
        \end{align*}
        \label{qci}
    \end{enumerate}
    \label{series of type M}
\end{definition}

\begin{remark}
    \begin{enumerate}
        \item Let \(\mathscr{F}=\{G(q)| q \in \mathscr{P}\}\) be a series of groups of type \(M\) with parameters \(\{c_i\}_{i \in I}\) then 
        \begin{align*}
            W(q)\simeq W(q'), \ s_i^{(q)} \mapsto s_i^{(q')}
        \end{align*}
        defines an isomorphism of groups. \\

        Thus for \(q_0:= \min(\mathscr{P})\) we can identify 
        \begin{align*}
           W(q) &\text{ with } W:=W(q_0) \\
           s_i^{q} &\text{ with } s_i:=s_i^{(q_0)} 
        \end{align*}
        (we introduce a new language game here that is context dependant see Wittgenstein)
        \begin{proof}
        Since \(\mathscr{F}\) is a series of groups, we know that Weyl groups \(W(q)=\langle S_q \rangle\) for \(q \in \mathscr{P}\) 
        are generated by a set \(S_q:=\{s_i^{(q)}\}_{i \in I} \). Now for any \(q' \in \mathscr{P}\) consider the map
        \begin{align*}
            f: S_q \rightarrow W(q'), \ s_i^{(q)} \mapsto s_i^{(q')}
        \end{align*}
        Then for \(i,j \in I, i \neq j\) we have \(m_{ij}=ord(s_i^{(q)}s_j^{(q)})=ord(s_i^{(q')}s_j^{(q')})\) and thus by 
        \ref{remark:commutativity of products} we know that 
        \begin{align*}
            f(s_i^{(q)})f(s_j^{(q)})f(s_i^{(q)})\dots = s_i^{(q')}s_j^{(q')}s_i^{(q')} \dots =s_j^{(q')}s_i^{(q')}s_j^{(q')} \dots = f(s_j^{(q)})f(s_i^{(q)})f(s_j^{(q)}) \dots
        \end{align*}
        Thus \(f\) satisfies condition \eqref{M} and we can use \ref{Matsumoto} and \ref{corollary:Matsumoto} to get a 
        group homomorphism \(\hat{f}:W(q) \rightarrow W(q')\) with \(f(s_i^{(q)})=s_i^{(q')}\). It is clearly surjective
        since it hits all generator of \(W(q')\) and since for \(w \in W\) with \(w = s_1^{(q)}\dots s_k^{(q)}\) reduced expression we have
        \begin{align*}
            1=f(w)=s_1^{(q')}\dots s_k^{(q')} \Leftrightarrow k = 0 \Leftrightarrow w = 1
        \end{align*}
        it is also injective. 
        \end{proof}
        \item We have \(q_{s_i} := ind(s_i) = q^{c_i} \)
        \begin{proof}
            By definition we have
            \begin{align*}
                q_w := ind(w):= \frac{|B(q) n^{(q)}_w B(q)|}{|B(q)|}
            \end{align*}
            where \(n^{(q)}_w\) is a representative of the class \(w \in W\) i.e. \(n^{(q)}_w \in w \in W=N(q)/H(q)\) i.e. \(n^{(q)}_w \in N(q)\)
            But by \ref{qci} in \ref{series of type M} this implies 
            \begin{align*}
                q_{s_i} := ind(s_i):= \frac{|B(q) n^{(q)}_{s_i} B(q)|}{|B(q)|} = q^{c_i}
            \end{align*}
            where the expression on the left hand side is a potence of the natural prime potence \(q\).
        \end{proof}
        \item Analogously as we have done before we can define the idempotent
        \begin{align*}
            e(q):= \frac{1}{|B(q)|} \sum_{b \in B(q)}^{} b \in A(q):=\C G(q)
        \end{align*}
        and Hecke algebras
        \begin{align*}
            H(q):= e(q) A(q) e(q)
        \end{align*}
        with basis \(\{T_w | w \in W\}\) where 
        \begin{align*}
            T_w := q_w e(q) n_w^{(q)} e(q)
        \end{align*}
        Multiplication of the basis elements is given by:
        \begin{align*}
            T_{s_i} T_w =
            \begin{cases}
                T_{s_i w}& \text{if} \quad l(s_i w) = l(w) +1 \\
                q^{c_i}T_{s_i w}+(q^{c_i}-1)T_w & \text{if} \quad l(s_i w)= l(w)-1 \\
            \end{cases}
        \end{align*}
    \end{enumerate}
\end{remark}

\begin{proposition}
    Let \(\mathscr{F}=\{G(q) | q \in \mathscr{P}\}\) be a series of groups of type \(M\) with parameters \(\{c_i\}_{i \in I}\) with 
    Weyl group \(W= \langle s_i | i \in I \rangle\). Let \(K = \Q(u)\) be the field of fractions of the polynom ring \(\Q[u]\) (\(u\): indeterminante).
    Let \(\mathscr{H}\) be a \(K\)-vs with basis \(\{T_w | w \in W\}\) (this can actually be any basis indexed by \(W\), you don't have to call them \(T_w\)). Then \(\mathscr{H}\) is an algebra with multiplication of the basis elements 
    given by
    \begin{align}
        T_{s_i} T_w =
        \begin{cases}
            T_{s_i w}& \text{if} \quad l(s_i w) = l(w) +1 \\
            u^{c_i}T_{s_i w}+(u^{c_i}-1)T_w & \text{if} \quad l(s_i w)= l(w)-1 \\
        \end{cases}
        \label{generatingDef}
    \end{align}
    for all \(i \in I, \ w \in W\). \(\mathscr{H}\) is called the \underline{generic Hecke algebra associated with \(\mathscr{F}\)}.
\end{proposition}
\begin{proof}
    Since we already know that \(\mathscr{H}\) is a \(K\)-vs we only need to define a multiplication on it. We will do this on the basis elements.
    For any \(i \in I\) define a linear map
    \begin{align}
        L_i: \mathscr{H} &\rightarrow \mathscr{H}, \quad L_i(T_w):=             
            \begin{cases}
                T_{s_i w}& \text{if} \quad l(s_i w) = l(w) +1 \\
                u^{c_i}T_{s_i w}+(u^{c_i}-1)T_w & \text{if} \quad l(s_i w)= l(w)-1 \\
            \end{cases}
        \label{L i first}
    \end{align}
    Now consider the monoid \(\mathcal{M}:= End_K(\mathscr{H})\) and the map \(f: S:= \{s_i | i \in I\} \rightarrow \mathcal{M}, \ s_i \mapsto L_i \)
    Now let \(i, j \in I, i \neq j\) and \(m:= ord(s_i s_j)\). To apply Matsumoto we must check that 
    \begin{align}
        L_i \circ L_j \circ L_i \circ \dots = L_j \circ L_i \circ L_j \circ \dots 
        \label{Matsumoto Condition}
    \end{align}
    where both terms have \(m\) factors. Let call the left map \(A\) and the right one \(B\). Since we have \(A, B \in End_K(\mathscr{H})\), 
    we can consider their representing matrix with respect to the basis \(\mathscr{B}= \{T_w | w \in W\}\) of \(\mathscr{H}\) given by
    \begin{align*}
        A(T_y)=\sum_{x \in W}^{} f_{xy} T_x \ , &&  B(T_y)=\sum_{x \in W}^{} g_{xy} T_x \ , &&y \in W
    \end{align*}
    Since \( u \in K\) recursively applying \(L_i\) and \(L_j\) gives \(f_{xy}, g_{xy} \in \Z[u], x,y \in W\). 
    We want to show now that \(f_{xy} = g_{xy}, x,y \in W\). 
    Select any \(q \in \mathscr{P}\), then \(\mathscr{B}(q):=\{T_w(q):=T_w| w \in W\}\) is a basis of \(H(q)\).
    Define analogous maps
    \begin{align}
        L_i:= L^q_i: H(q) &\rightarrow H(q), \quad L_i(T_w):=             
            \begin{cases}
                T_{s_i w}& \text{if} \quad l(s_i w) = l(w) +1 \\
                q^{c_i}T_{s_i w}+(q^{c_i}-1)T_w & \text{if} \quad l(s_i w)= l(w)-1 \\
            \end{cases}
        \label{L i second}
    \end{align}
    Then we have
    \begin{align}
        A^q:= L_i\circ L_j \circ L_i \circ \dots = L_j \circ L_i \circ L_j  \dots = B^q 
        \label{Aq=Bq}
    \end{align}
    since 
    \begin{align*}
        A^q(T_w)=T_{s_i}T_{s_j}T_{s_i}\dots T_w = T_{s_j}T_{s_i}T_{s_j} \dots T_w = B^q(T_w)
    \end{align*}
    by 2.8. Now since \eqref{L i second} comes from \eqref{L i first} just by replacing the \(u \mapsto q\) also the representing matrices 
    resemble each other via
    \begin{align*}
        A^q(T_y)=\sum_{x \in W}^{} f_{xy}(q) T_x \ , &&  B^q(T_y)=\sum_{x \in W}^{} g_{xy}(q) T_x \ , &&y \in W
    \end{align*}
    But by \eqref{Aq=Bq} we know that \(f_{xy}(q)=g_{xy}(q), \ x,y \in W\) But since this is true for every \(q \in \mathscr{P}\) 
    and \(\mathscr{P}\) contains infinitly many elements we also have that \(f_{xy}=g_{xy}, \ x,y \in W\), what is what we wanted to show.
    Thus we have prooven \eqref{Matsumoto Condition} and Matsumoto gives us a linear map \(L_w \in End_K(\mathscr{H})\) for any \(w \in W\)
    such that
    \begin{align*}
        L_w = L_{s_1} \circ \dots \circ L_{s_r}
    \end{align*}
    for any reduces expression \(w = s_1 \dots s_r\). Thus we can define the product in \(\mathscr{H}\) via
    \begin{align*}
        T_w \cdot T_{w'} := L_w(T_{w'})
    \end{align*}
    Like above we can express the product through the basis \(\mathscr{B}\) via
    \begin{align*}
        T_w \cdot T_{w'} = \sum_{x \in W}^{} h_{ww'x}T_x, \quad h_{ww'x} \in \Z[u]
    \end{align*}
    We need to show now that this product is associative. But again the product works the same way in 
    \(H(q)\) for any \(q \in \mathscr{P}\) with the basis \(\mathscr{B}(q)\).
    Thus the values of the polynomials occuring in 
    \begin{align}
        \begin{split}
            T_{w_1}(T_{w_2}T_{w_3})&=T_{w_1} \left(\sum_{x \in W}^{} h_{w_2w_3x}T_x\right) \\
            &= \sum_{x \in W}^{} h_{w_2w_3x} T_{w_1}T_x \\
            &= \sum_{x \in W}^{}\sum_{y \in W}^{} h_{w_2w_3x} h_{w_1xy}T_y \\
            &= \sum_{y \in W}^{}\left(\sum_{x \in W}^{} h_{w_2w_3x} h_{w_1xy}\right)T_y
        \end{split}
        \label{associativity computation}
    \end{align}
    and 
    \begin{align*}
        (T_{w_1}T_{w_2})T_{w_3} = \sum_{y \in W}^{}\left(\sum_{x \in W}^{} h_{w_1w_2x} h_{w_3xy}\right)T_y
    \end{align*}
    have to be equal when applied to \(q\) i.e.\(\sum_{x \in W}^{}h_{w_2w_3x}(q) h_{w_1xy}(q) = \sum_{x \in W}^{}h_{w_1w_2x}(q) h_{w_3xy}(q)\).
    But like above this implies \(\sum_{x \in W}^{}h_{w_2w_3x} h_{w_1xy} = \sum_{x \in W}^{}h_{w_1w_2x} h_{w_3xy}\), which in turn makes the
    product associative on the basis and thus also on all elements. 
\end{proof}

\begin{remark}
    If we take the polynomials \(h_{xyz} \in \Z[u]\) generated by \eqref{generatingDef} then we have 
    \begin{align*}
        T_x T_y = \sum_{z \in W}^{}h_{xyz}T_z \, \quad x,y \in W
    \end{align*}
    Now fix \(\alpha \in \C\) and let \(\mathscr{H}_\alpha\) be a \(\C\)-vs with a basis 
    \(\mathscr{B}_\alpha:=\{T^\alpha_w | w \in W\}\) indexed by \(W\) (again we could give it any name: \(\{e_w | w \in W\}\))
    then we can define a product 
    \begin{align*}
        T^\alpha_x T^\alpha_y = \sum_{z \in W}^{}h_{xyz}(\alpha)T^\alpha_z \, \quad x,y \in W
    \end{align*}
    which by the calculation in \eqref{associativity computation} this product is associative. 
    We can not define the multiplication via
    \begin{align*}
        T^\alpha_x T^\alpha_y = \sum_{z \in W}^{}h_{xyz}T^\alpha_z \, \quad x,y \in W
    \end{align*}
    since \(h_{xyz} \in \Z[u]\). 

    \begin{itemize}
        \item[\(\alpha = 1\):] By \eqref{generatingDef} we know that
        \begin{align*}
            T^\alpha_{s_i} T^\alpha_w = T^\alpha_{s_i w} \ , \quad i \in I, w \in W
        \end{align*}
        Thus multiplication works the same way as in \(\C W\).
        \item[\(\alpha = q \in \mathscr{P}\):] Then the product
        \begin{align*}
            T^q_{s_i} T^q_w =
            \begin{cases}
                T^q_{s_i w}& \text{if} \quad l(s_i w) = l(w) +1 \\
                q^{c_i}T^q_{s_i w}+(q^{c_i}-1)T^q_w & \text{if} \quad l(s_i w)= l(w)-1 \\
            \end{cases}
        \end{align*}
        of the basis \(\mathscr{B}_\alpha\) works the same way as for the basis \(\mathscr{B}= \{T_w | w \in W\}\) of \(H(q)\).
        Thus we have \( \mathscr{H}_q\simeq H(q)\)
    \end{itemize}
\end{remark}

\section{The Kashdan-Lusztig basis of \(\mathscr{H}\)}

Now let \(\mathscr{H}\) be the free module over the Laurent polynomial ring \(R:=\Z[v,v^{-1}]\) with indeterminante \(v\) and basis \(W\)
i.e. \(\mathscr{H}:=R^{(W)}:= \bigoplus_{w \in W} R \). Let denote the basis elements \(w\) again by \(T_w\), then we have
\begin{align*}
    \mathscr{H}= \left\{\sum_{w \in W}^{}a_w T_w \middle| a_w \in R\right\}
\end{align*}
Now define an associative multiplication on \(\mathscr{H}\) via
\begin{align*}
    T_x T_y = \sum_{z \in W}^{}h_{xyz}(v^2)T_z \, \quad x,y \in W
\end{align*}
Furthermore assume that \(c_i = 1, i \in I\) ( for example this holds in \(G:= GL_n(\F_q), \ W \simeq S_n\) see example below 3.7).
If we define
\begin{align*}
    \widetilde{T_w}:=v^{-l(w)}T_w
\end{align*}
we still have a basis \(\{\widetilde{T_w} | w \in W\}\) of \(\mathscr{H}\), where the product is given by
\begin{align*}
    \widetilde{T}_{s} \widetilde{T}_w =
    \begin{cases}
        \widetilde{T}_{s w}& \text{if} \quad l(s w) = l(w) +1 \\
        \widetilde{T}_{s w}+(v-v^{-1})\widetilde{T}_w & \text{if} \quad l(s w)= l(w)-1 \\
    \end{cases}
\end{align*}

Next we have a ring homomorphism 
\begin{align*}
    \overline{\phantom{x}}: \Z[v,v^{-1}] &\rightarrow \Z[v,v^{-1}] \\
    v &\mapsto v^{-1}
\end{align*}

and we know that \(T_w, w \in W\) has in inverse, since the inverse of \(T_s\) is given by
\begin{align*}
    T_s^{-1}=u^{-1}T_s-(1-u^{-1})T_1
\end{align*}
Thus also \(\widetilde{T}_w\) has an inverse since the inverse of \(\widetilde{T}_s\) is given by
\begin{align*}
    \widetilde{T}_s^{-1} = v T_s^{-1}
\end{align*}

\begin{lemma}
    The extension of the abouve map onto \(\mathscr{H}\) which is given by:
    \begin{align*}
        \overline{\phantom{x}}:\mathscr{H} &\rightarrow \mathscr{H}, \ \overline{\sum_{w \in W}^{}a_w \widetilde{T}_w} := \sum_{w \in W}^{} \overline{a_w} \ \widetilde{T}_w^{-1} 
    \end{align*}
    is a ring homomorphism. 
\end{lemma}






%%%%%%%%%%%%%%%%%%%%%%%%%%%%%%%%%%%%%%%%%%%%%%%%%%%%%%%%%%%%%%%%%%%%%%%%%%%%%%%%%%%
\begin{thebibliography}{9}
\bibitem{jan}
Jantzen, Schwermer
\textit{Algebra}
\bibitem{bosch}
Bosch - Lineare Algebra
\textit{Algebra}
\bibitem{kehr}
Kehrer, Maximilian
\textit{Skript}
\end{thebibliography}
\end{document}